\documentclass{mi-seminar}

\usepackage{hyperref}

\title{Flying Etiquette}
%\subtitle{Dokumentation zum Projekt Informationsvisualisierung}
\author{Kevin Angermeyer \\ \and Lydia Güntner \\ \and Maximiliane Windl}



\title{Flying Etiquette}
\author{Kevin Angermeyer, Lydia Güntner, Maximiliane Windl}
\semester{SS18}
\course{Projektseminar Mediengestaltung I: Informationsvisualisierung}
\module{MEI-M05.3 }
\dozent{Florin Schwappach}
\studid{[Matrikelnummer Kevin], 1785088, 1770307}
\studSemester{6. Semester B.A. Medieninformatik / Informationswissenschaft}


%\address{Domplatz 1, 93047 Regensburg}{} % Optional
\mail{kevin.angermeyer@stud.uni-regensburg.de, lydia-maria.guentner@stud.uni-regensburg.de, maximiliane.windl@stud.uni-regensburg.de}




\begin{document}

\maketitle

\section{Das Projekt starten} \label{Das Projekt starten}
\subsection{Unix}
\textbf{Was wird benötigt?}\newline
Python\newline\newline
\textbf{Wie wird das Projekt gestartet?}\newline
1. Über das Terminal zum Projekt navigieren\newline
2. Im Projekt durch den Befehl\newline 
\textit{python -m SimpleHTTPServer}\newline
den Server starten.\newline
3. im Browser auf localhost:8000 wechseln.

\section{Datensatz} \label{Datensatz}
Bei dem Datensatz handelt es sich um den Datensatz \textit{flying-etiquette-survey} von \textit{\href{https://github.com/fivethirtyeight/data/tree/master/flying-etiquette-survey}{fivethirtyeight}}. Die Daten stammen aus einer Online-Umfrage, welche mit \href{https://www.surveymonkey.com/mp/audience/}{SurveyMonkey}, am 29. und 30. August 2014 durchgeführt wurde. Insgesamt nahmen 1040 Probanden an der Studie teil. Nach einer Säuberung der Daten, bei der alle Teilnehmer die angaben nie zu fliegen und daher die anderen Fragen zum Thema Fliegen nicht beantwortet hatten, entfernt wurden, blieben Daten von 856 Probanden übrig. 
\newline \newline
Anschließend wurde der Datensatz in verschiedene Kategorien aufgeteilt, denen folgende Fragen zugeordnet wurden:
\begin{enumerate}

\item demographischen Fragen:
\begin{itemize}
\item Gender
\item Age
\item Household income
\item Education 
\item Location
\item How tall are you? 
\item How often do you travel by plane? 
\newline \newline
\end{itemize}


\item auf Kinder bezogene Fragen:
\begin{itemize}
\item Do you have any children under 18?
\item In general, is it rude to bring a baby on a plane?
\item In general, is it rude to knowingly bring unruly children on a plane? 
\newline \newline
\end{itemize}


\item auf das Zurücklehnen der Sitze bezogene Fragen: 
\begin{itemize}
\item Is it rude to recline your seat on a plane? 
\item Under normal circumstances, does a person who reclines their seat during a flight have any obligation to the person sitting behind them? 
\item Given the opportunity, would you eliminate the possibility of reclining seats on planes entirely?
\newline \newline
\end{itemize}


\item auf den Sitzplatz bezogene Fragen: 
\begin{itemize}
\item In a row of three seats, who should get to use the two arm rests? 
\item In a row of two seats, who should get to use the middle arm rest?
\item Who should have control over the window shade?
\newline \newline
\end{itemize}


\item auf das Tauschen von Sitzplätzen bezogene Fragen: 
\begin{itemize}
\item Is it rude to ask someone to switch seats with you in order to be closer to friends?
\item Is it rude to ask someone to switch seats with you in order to be closer to family?
\item Is it rude to move to an unsold seat on a plane? 
\newline \newline
\end{itemize}


\item auf die Kommunikation mit anderen Fluggästen bezogene Fragen:
\begin{itemize}
\item Generally speaking, is it rude to say more than a few words to the stranger sitting next to you on a plane?
\item On a 6 hour flight from NYC to LA, how many times is it acceptable to get up if you're not in an aisle seat?
\item Is it rude to wake a passinger up if you are trying to go to the bathroom?
\item Is it rude to wake a passenger up if you are trying to walk around?
\newline \newline
\end{itemize}


\item auf im Flugzeug verbotene Dinge bezogene Fragen:
\begin{itemize}
\item Have you ever used personal electronics during take off or landing in violation of a flight attendant's direction?
\item Have you ever smoked a cigarette in an airplane bathroom when it was against the rules? 
\end{itemize}


\end{enumerate}

\section{Startdiagramm}

\section{Bubblediagramme}
Nachdem das Startdiagramm einen Überblick über die Daten geben sollte, geben die Bubblediagramme die Möglichkeit die in \ref{Datensatz} beschriebenen Kategorien genauer und interaktiv zu erforschen.
Die Wahl fiel auf Bubblediagramme, da der Nutzer hier auf einen Blick die Verhältnisse der gegebenen Antworten sehen kann. Die Größe der Bubbles spiegelt nämlich immer die Anzahl der gegeben Antwort wider. Über den Bubbles befinden sich alle möglichen Antworten. Diese sind durch Linien mit der zugehörigen Bubble verbunden. Wenn der Nutzer über die Bubbles hovert, erscheint auf diesen die Anzahl der gegebenen Antworten und die Anzahl aller Antworten. So steht dem Nutzer neben der visuellen Größe der Bubbles, noch ein Zahlenwert zur Verfügung. 

Für jede Kategorie gibt es eine eigene Unterseite. Auf diesen befinden sich jeweils zwei bis drei Bubblediagramme, welche die Fragen der Kategorien repräsentieren. Jede Bubble steht für eine Antwort, die auf die jeweilige Frage geben werden konnte. 

\subsection{Filter}
Die Bubblediagramme können mit Hilfe von Filtern verändert werden. Diese befinden sich jeweils rechts der Diagramme. Die Filter sind durch Icons und im Falle der Dropdown-Filter, durch einen kurzen Text gekennzeichnet. Die Wahl fiel auf Icons, da der Nutzer so sofort die Funktionalität des Filters erfassen kann. Ist das Icon aber trotzdem nicht eindeutig genug, kann der Nutzer zusätzlich über die Filter hovern und erhält dann eine genauere Erklärung, für was der jeweilige Filter steht. 

Werden die Filter gesetzt, passt sich die Größe der Bubbles an die Anzahl der gegebenen Antworten an. Das heißt die Bubbles sind anteilig immer so groß, wie die Anzahl der gegebenen Antworten. Das stellt sicher, dass der Nutzer auf einen Blick die Verhältnisse wahrnehmen kann.  

Es wird zwischen Standardfilter und speziellen Filtern, die nicht bei allen Diagrammen zur Verfügung stehen, unterschieden. Die Entscheidung spezielle Filter einzuführen fiel, da bestimmte Filter nur bei manchen Fragen Sinn machen. So macht es beispielsweise nur Sinn die Größe der befragten Person mit einzubeziehen, wenn absehbar ist, dass die Antwort auf eine Frage durch die Größe beeinflusst wird. So beispielsweise bei der Frage, wie unhöflich es ist, den Sitz im Flugzeug zurückzulehnen. Man kann davon ausgehen, dass es für größere Personen mit längeren Beinen wesentlich unangenehmer ist, wenn die Person vor ihm den Sitz zurücklehnt. Deshalb kann es sein, dass die Antwort durch die Größe beeinflusst wird.

Folgende Standardfilter stehen bei allen Diagrammen zur Verfügung:
\begin{itemize}
\item Gender-Filter: Mit dem Gender-Filter kann nach dem Geschlecht, also nach männlich oder weiblich gefiltert werden.
\item FrequencyOfAirTravel-Filter: Mit dem FrequencyOfAirTravel-Filter kann nach der Häufigkeit, mit der die Testpersonen fliegen gefiltert werden. Zur Auswahl stehen: 
	\begin{itemize}
	\item Once a year or less
	\item Once a month or less
	\item A few times per month
	\item A few times per week
	\item Every day
	\end{itemize}
\item Age-Filter: Mit dem Age-Filter kann nach dem Alter der Testpersonen gefiltert werden. Zu Auswahl stehen: 
	\begin{itemize}
	\item 18 - 29
	\item 30 - 44
	\item 45 - 60
	\item > 60
	\end{itemize}
\item Income-Filter: Mit dem Income-Filter kann nach dem Jahreseinkommen der Testpersonen gefiltert werden. Zur Auswahl stehen: 
	\begin{itemize}
	\item \$0 - \$24,999
	\item \$25,000 - \$49,999
	\item \$50,000 - \$99,999
	\item \$100,000 - \$149,999
	\item \$150000
	\end{itemize}
\item Education-Filter: Mit dem Education-Filter kann nach dem höchsten Bildungsabschluss der Testpersonen gefiltert werden: Zur Auswahl stehen:
	\begin{itemize}
	\item High school degree
	\item Some college or Associate degree
	\item Graduate degree
	\item Bachelor degree
	\end{itemize}
\item Location-Filter: Mit dem Location-Filter kann nach der Region der USA, aus der die Testpersonen kommen gefiltert werden. Zur Auswahl stehen:
	\begin{itemize}
	\item New England
	\item East North Central
	\item East South Central
	\item West North Central
	\item West South Central
	\item Middle Atlantic
	\item South Atlantic
	\item Mountain
	\item Pacific
	\end{itemize}
\end{itemize}

Zusätzlich zu den Standardfiltern stehen auf zwei Unterseiten Sonderfilter zur Verfügung. 

Auf der Unterseite \textit{Children} ein Child-Filter, mit dem danach gefiltert werden kann, ob die Testpersonen mindestens ein Kind unter 18 Jahren haben. 

Auf der Unterseite \textit{Seat Reclining} stehen folgende zwei Sonderfilter zur Verfügung: 
\begin{itemize}
\item FrequencyOfeatReclining-Filter: Mit dem FrequencyOfeatReclining-Filter kann nach der Häufigkeit mit der die Testpersonen ihren Sitz zurücklehnen gefiltert werden. Zur Auswahl stehen:
	\begin{itemize}
	\item Never
	\item Once in a while
	\item About half the time
	\item Usually
	\item Always
	\end{itemize}
\item Height-Filter: Mit dem Height-Filter kann nach der Größe der Testpersonen gefiltert werden. Dabei können auf einem Slider Größen vom 5 Fuß bis größer als 6 Fuß und 6 Zoll gesetzt werden.
\end{itemize} 

\section{Aufteilung}


\end{document}
